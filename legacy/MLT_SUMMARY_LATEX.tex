% MLT Alpha Calibration - Methodology Summary
% Ready to copy-paste into Overleaf

\subsection{Mixing-Length Theory Calibration}

We calibrate the mixing-length parameter $\alpha$ for adiabatic atmospheric layers using standard mixing-length theory (MLT). For each temperature-pressure combination $(T, P)$, representing the center of a convective layer, we determine the $\alpha$ value that allows MLT to transport the required convective flux while maintaining a nearly adiabatic temperature gradient.

\subsubsection{Key Equations}

\textbf{Pressure scale height:}
\begin{equation}
H_p = \frac{RT}{\mu g}
\end{equation}
where $R = 8.314$ J/(mol$\cdot$K) is the universal gas constant, $\mu$ is the mean molecular weight, and $g$ is surface gravity.

\textbf{Mixing length:}
\begin{equation}
\ell = \alpha H_p
\end{equation}
where $\alpha$ is the dimensionless mixing-length parameter.

\textbf{Adiabatic gradient:}
\begin{equation}
\nabla_{\text{ad}} = \frac{R}{\mu c_p}
\end{equation}
where $c_p$ is the specific heat capacity at constant pressure.

\textbf{Assumed temperature gradient:}
\begin{equation}
\nabla = \nabla_{\text{ad}} \times (1 + \epsilon)
\end{equation}
with $\epsilon = 0.001$ (0.1\% superadiabatic), required to drive convection.

\textbf{MLT convective flux:}
\begin{equation}
F_c(\alpha) = \rho \, c_p \, T \, S(\alpha) \, (\nabla - \nabla_{\text{ad}})^{3/2}
\end{equation}
where $\rho$ is density and
\begin{equation}
S(\alpha) = \sqrt{\frac{g\delta}{8H_p}} \, \ell^2 = \sqrt{\frac{g\delta}{8H_p}} \, (\alpha H_p)^2
\end{equation}
with $\delta = 1$ for an ideal gas.

\textbf{Calibration condition:}
\begin{equation}
F_c(\alpha) = F_{\text{conv}}
\end{equation}
We solve this equation for $\alpha$, where $F_{\text{conv}}$ is the convective flux requirement for each atmospheric regime.

\subsubsection{Parameter Space}

We explore $T \in [200, 2000]$ K and $P \in [10^{-5}, 10^3]$ bar using a 20$\times$20 grid (linearly spaced in $T$, logarithmically in $P$). Three planetary regimes are considered:

\begin{itemize}
    \item \textbf{Hot Jupiter:} $\mu = 0.0022$ kg/mol (H$_2$), $c_p = 14000$ J/(kg$\cdot$K), $g = 10$ m/s$^2$, $F_{\text{conv}} = 5 \times 10^6$ W/m$^2$
    \item \textbf{Sub-Neptune:} $\mu = 0.0035$ kg/mol (H$_2$/He), $c_p = 10000$ J/(kg$\cdot$K), $g = 15$ m/s$^2$, $F_{\text{conv}} = 1 \times 10^6$ W/m$^2$
    \item \textbf{Terrestrial:} $\mu = 0.029$ kg/mol (N$_2$/O$_2$), $c_p = 1005$ J/(kg$\cdot$K), $g = 9.81$ m/s$^2$, $F_{\text{conv}} = 200$ W/m$^2$
\end{itemize}

Results show $\alpha$ varies from $\sim$0.0001 to $\sim$100 across the parameter space, with terrestrial atmospheres exhibiting systematically smaller $\alpha$ values (median $\alpha = 0.015$) compared to gas giants (median $\alpha \sim 0.3$).


